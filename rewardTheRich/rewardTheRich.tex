\documentclass[11pt,a4paper]{report}
\usepackage{amsmath,amsfonts,amssymb,amsthm,epsfig,epstopdf,titling,url,array}
\usepackage{enumitem}
\usepackage{changepage}
\usepackage{graphicx}
\usepackage{caption}
\theoremstyle{plain}
\newtheorem{thm}{Theorem}[section]
\newtheorem{prop}[thm]{Proposition}
\newtheorem*{cor}{Corollary}
\theoremstyle{definition}
\newtheorem{defn}{Definition}[section]
\newtheorem{conj}{Conjecture}[section]
\newtheorem{exmp}{Example}[section]
\newtheorem{exercise}{Exercise}[section]
\theoremstyle{remark}
\newtheorem*{rem}{Remark}
\newtheorem*{lem}{Lemma}
\newtheorem*{note}{Note}
\def\changemargin#1#2{\list{}{\rightmargin#2\leftmargin#1}\item[]}
\let\endchangemargin=\endlist
\begin{document}


\section*{Problem}
Let $x_0 < x_1 < … < x_{n-1}$ and $y_0 < y_1 … < y_{n - 1}$ be strictly increasing sequences of positive numbers.  Show that for any permutation $\sigma$ of $\{0, … , n-1\}$,
$$\sum_{i = 0}^{n-1} x_i y_{\sigma(i)}< \sum_{i = 0}^{n-1} x_i y_i$$
unless $\sigma$ is the identity permutation.  What this says is that when forming dot products of two vectors of distinct values, the product is maximized when the entries appear in the same order.

\section*{Solution}
 Let $\sigma$ be an arbitrary permutation of  $\{0, … , n-1\}$ and define $f(\sigma) = \sum_{i = 0}^{n-1} x_i y_{\sigma(i)}$.   It suffices to show that if $\sigma$ is not the identity permutation, then there is another permutation $\tau$ such that $f(\tau) > f(\sigma)$.    This is sufficient because there are only finitely many permutations of $\{0, … , n-1\}$ so $f$ must have a maximum that is attained by some permutation.  What we are showing is that no non-identity permutation can attain the maximum.
 \\\\
 We start by establishing a lemma.
 \lem  If $x_1< x_2$ and $y_1 < y_2$ are positive real numbers then  $x_1 y_1 + x_2 y_2> x_1 y_2 + x_2 y_1$.
 \proof Write $x_2 = x_1 + a$ and $y_2 = y_1 + b$ where $a$ and $b$ are positive.  Then $x_1 y_1 + x_2 y_2 = x_1 y_1 + (x_1 + a)(y_1 + b) = 2x_1y_1 + x_1 b + y_1a + ab$ and $x_1 y_2 + x_2 y_1 = x_1(y_1 + b) + (x_1 + a)y_1 = 2x_1y_1 + x_1b + y_1a,$ so since $a$ and $b$ are positive $ab$ is positve and the inequality is proved.
 \\\\
 Now consider an arbitrary permutation $\sigma$ of $\{0, … , n-1\}$.  If $\sigma$ is not the identity, let $i$ be the largest index such that $\sigma(i)  \ne i$. So for all $k > i$, $\sigma(k) = k$.  Since permutations are 1-1 mappings, we must have $\sigma(i) < i$ (all values above $i$ are taken by the fixed points above $i$).  Let $j = \sigma(i)$ and let $l$ be the pre-image of $i$  under $\sigma$,  so $\sigma(l) = i$.  By the lemma, if we modify $\sigma$ to make $\sigma(i) = i$ and $\sigma(l) = j$, the portion of $\sum_{i = 0}^{n-1} x_i y_{\sigma(i)}$ contributed by the indexes $i$ and $l$ will increase while the rest of the terms remain the same.  So making $\tau$ the permuation that modifies $\sigma$ in this way, we have shown what we need to show to establish the main result.






\end{document}

