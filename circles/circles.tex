\documentclass[11pt,a4paper]{report}
\usepackage{amsmath,amsfonts,amssymb,amsthm,epsfig,epstopdf,titling,url,array}
\usepackage{enumitem}
\usepackage{changepage}
\usepackage{graphicx}
\usepackage{caption}
\theoremstyle{plain}
\newtheorem{thm}{Theorem}[section]
\newtheorem{lem}[thm]{Lemma}
\newtheorem{prop}[thm]{Proposition}
\newtheorem*{cor}{Corollary}
\theoremstyle{definition}
\newtheorem{defn}{Definition}[section]
\newtheorem{conj}{Conjecture}[section]
\newtheorem{exmp}{Example}[section]
\newtheorem{exercise}{Exercise}[section]
\theoremstyle{remark}
\newtheorem*{rem}{Remark}
\newtheorem*{note}{Note}
\def\changemargin#1#2{\list{}{\rightmargin#2\leftmargin#1}\item[]}
\let\endchangemargin=\endlist 
\begin{document}


\section*{Problem} Given a circle of radius $r$, write a function $f(r)$ that
gives the radius of a circle whose area is the square of the area of the
original circle.

\section*{Bonus} Start with a circle of radius $10$ and draw another circle of
radius $r$ inside it.  What value of $r$ makes the product of the of the areas
of the inner and outer circles largest?

\section*{Solution} The square of the area of a circle of radius r is
$( \pi r^2)^2 = \pi^2r^4$.   So if we make $f(r) = \sqrt{\pi}r^2$ then a circle
of this radius will have area $\pi ( \sqrt{\pi}r^2)^2 = \pi^2r^4$ as desired.

\section*{Bonus Solution}
 
 
\end{document}

