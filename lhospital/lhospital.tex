\documentclass[11pt,a4paper]{report}
\usepackage{amsmath,amsfonts,amssymb,amsthm,epsfig,epstopdf,titling,url,array}
\usepackage{enumitem}
\usepackage{changepage}
\usepackage{graphicx}
\usepackage{caption}
\theoremstyle{plain}
\newtheorem{thm}{Theorem}[section]
\newtheorem{lem}[thm]{Lemma}
\newtheorem{prop}[thm]{Proposition}
\newtheorem*{cor}{Corollary}
\theoremstyle{definition}
\newtheorem{defn}{Definition}[section]
\newtheorem{conj}{Conjecture}[section]
\newtheorem{exmp}{Example}[section]
\newtheorem{exercise}{Exercise}[section]
\theoremstyle{remark}
\newtheorem*{rem}{Remark}
\newtheorem*{note}{Note}
\def\changemargin#1#2{\list{}{\rightmargin#2\leftmargin#1}\item[]}
\let\endchangemargin=\endlist 
\begin{document}


\section*{Problem}
Evaluate $lim_{x\to 0} sin(x)/x$.

\section*{Solution}
This can be evaluated using l'Hospital's rule, which states that if $f$ and $g$ are continuously differentiable at $a$, and  $\lim_{x \to a} f(x) = 0 =  \lim_{x \to a} g(x)$ then $\lim_{x \to a} {f(x)/g(x)} = f'(0) / g'(0)$.  So $lim_{x\to 0} sin(x)/x = \lim_{x \to 0} {cos(0) / 1} = 1.$

But why is l'Hospital's rule true?  This follows nicely from the definition of the derivative.  Recall that the derivative of $f$ at $x$ is defined by
$$f'(x) = \lim_{h \to 0} \frac{f(x + h) - f(x)}{h}.$$
So $$\frac{f'(0)}{g'(0)} = $$ $$\frac{\mathop{\lim}\limits_{h \to 0} \frac {f(0 + h) - f(0)} {h}} {\mathop{\lim}\limits_{h \to 0} \frac {g(0 + h) - g(0)} {h}} = $$ $$\lim_{h \to 0} \frac {f(h) - f(0)} {g(h) - g(0)}$$.
\end{document}