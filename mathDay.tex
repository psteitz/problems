\documentclass[11pt,a4paper]{report}
\usepackage{amsmath,amsfonts,amssymb,amsthm,epsfig,epstopdf,titling,url,array}
\usepackage{enumitem}
\usepackage{changepage}
\usepackage{graphicx}
\usepackage{caption}
\usepackage{listings}
\usepackage{color}
\theoremstyle{plain}
\newtheorem{thm}{Theorem}[section]
\newtheorem{lem}[thm]{Lemma}
\newtheorem{prop}[thm]{Proposition}
\newtheorem*{cor}{Corollary}
\theoremstyle{definition}
\newtheorem{defn}{Definition}[section]
\newtheorem{conj}{Conjecture}[section]
\newtheorem{exmp}{Example}[section]
\newtheorem{exercise}{Exercise}[section]
\theoremstyle{remark}
\newtheorem*{rem}{Remark}
\newtheorem*{note}{Note}
\def\changemargin#1#2{\list{}{\rightmargin#2\leftmargin#1}\item[]}
\let\endchangemargin=\endlist 

\definecolor{codegreen}{rgb}{0,0.6,0}
\definecolor{codegray}{rgb}{0.5,0.5,0.5}
\definecolor{codepurple}{rgb}{0.58,0,0.82}
\definecolor{backcolour}{rgb}{0.95,0.95,0.92}

\lstdefinestyle{mystyle}{
	backgroundcolor=\color{backcolour},   
	commentstyle=\color{codegreen},
	keywordstyle=\color{magenta},
	numberstyle=\tiny\color{codegray},
	stringstyle=\color{codepurple},
	basicstyle=\footnotesize,
	breakatwhitespace=false,         
	breaklines=true,                 
	captionpos=b,                    
	keepspaces=true,                 
	numbers=left,                    
	numbersep=5pt,                  
	showspaces=false,                
	showstringspaces=false,
	showtabs=false,                  
	tabsize=2
}

\lstset{style=mystyle}
\begin{document}
	
	\section*{Problem 1}
	(a) Find the smallest positive  integer that has exactling  143 proper factors.  An integer $m$ is proper factor or another integer $n$ if $1 < m < n$ and there is $n = m \times k$ for some positive integer $k$.  So for example, $12$ has proper factors $2,3,4$ and $6$.
	\\
	(b) Show that for every $n$ there are infinitely many integers with exactly $n$ proper factors.
	
	\section*{Problem 2}
	Find a sequence of real numbers from the interval ${0,1]$ that has subsequences converging to infinitely many different values.   For example, the secuence ${1/2 + 1/3, 1/3 + 1/3, 1/2 + 1/4, 1/3 + 1/4, 1/2 + 1/5, 1/3 + 1/5 ...}$ has two convergent subsequences , the first consisteing of the even numbered terms converges to 1/2 and the one made up of the odd-numbered terms converges to 1/3.
	
\end{document}
