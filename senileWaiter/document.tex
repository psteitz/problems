\documentclass[11pt,a4paper]{report}
\usepackage{amsmath,amsfonts,amssymb,amsthm,epsfig,epstopdf,titling,url,array}
\usepackage{enumitem}
\usepackage{changepage}
\usepackage{graphicx}
\usepackage{caption}
\usepackage{listings}
\usepackage{color}
\usepackage[utf8]{inputenc}
\usepackage[english]{babel}
\usepackage{hyperref}
\hypersetup{
	colorlinks=true,
	linkcolor=blue,
	filecolor=magenta,      
	urlcolor=cyan,
}
\urlstyle{same}
\theoremstyle{plain}
\newtheorem{thm}{Theorem}[section]
\newtheorem{lem}[thm]{Lemma}
\newtheorem{prop}[thm]{Proposition}
\newtheorem*{cor}{Corollary}
\theoremstyle{definition}
\newtheorem{defn}{Definition}[section]
\newtheorem{conj}{Conjecture}[section]
\newtheorem{exmp}{Example}[section]
\newtheorem{exercise}{Exercise}[section]
\theoremstyle{remark}
\newtheorem*{rem}{Remark}
\newtheorem*{note}{Note}
\def\changemargin#1#2{\list{}{\rightmargin#2\leftmargin#1}\item[]}
\let\endchangemargin=\endlist 

\definecolor{codegreen}{rgb}{0,0.6,0}
\definecolor{codegray}{rgb}{0.5,0.5,0.5}
\definecolor{codepurple}{rgb}{0.58,0,0.82}
\definecolor{backcolour}{rgb}{0.95,0.95,0.92}

\lstdefinestyle{mystyle}{
	backgroundcolor=\color{backcolour},   
	commentstyle=\color{codegreen},
	keywordstyle=\color{magenta},
	numberstyle=\tiny\color{codegray},
	stringstyle=\color{codepurple},
	basicstyle=\footnotesize,
	breakatwhitespace=false,         
	breaklines=true,                 
	captionpos=b,                    
	keepspaces=true,                 
	numbers=left,                    
	numbersep=5pt,                  
	showspaces=false,                
	showstringspaces=false,
	showtabs=false,                  
	tabsize=2
}

\lstset{style=mystyle}
\begin{document}

\section*{Senile Waiter Problem}
Suppose that 4 people go out to dinner together and they order 4 different meals from a nice but senile waiter who remembers and orders the correct meals but then distributes them randomly.  What is the probability that noone gets the meal that they ordered?

\section*{bonus}
Generalized your solution to $n$ diners - i.e., what is the probability that a random permutation of $n$ has no fixed points?

\section*{Solution}
Label the meals $0,1,2,3$.  Consider rearranged sequences as ways the waiter can distribute the meals.  For example, $(1,0,3,2)$ represents the delivery where the waiter switches diner 0 and diner 1's meals and does the same for 2 and 3.  There are $4\times3\times2\times1= 24$ ways to rearrange $0,1,2,3$.  We need to count how many of these don't leave any of the original digits in their original place.  In mathematical terms, we need to count $d_4$ = the number of permutations of $4 = \{0, 1, 2, 3\}$ that have no fixed points.  The swapping example above is one such permutation. 
\\\\
Now $d_4$ equals  24 minus the number of permutations that have fixed points. We count the complement by partitioning on the number of fixed points.  For $n = 1, 2, 3$ let $c_n$ be the number of permutations of 4 that have exactly $n$ fixed points.  Then the number the number of permutations of $4$ with no fixed points equals $$ 24 - \sum_{i=1}^{3}{c_i}$$
Now we compute the $c_i$.  Consider $c_{1}$.  We can count this by partitioning on the element that is fixed and recursing on $n$.  There are $ \binom{4}{1} = 4$ choices for the fixed point.  The permutations that fix just one point are extensions of permutations of $n - 1 = 3$ with no fixed points.  So if we define $d_n$ to be the number of permutations of $n$ that have no fixed points, then 
$$c_{1} =  \binom{4}{1} d_{3}$$
$$c_{2} = \binom{4}{2} d_{2}$$
$$c_{3} = \binom{4}{3} d_{1} = 4 \times 0 = 0$$
$$c_{4} = 1$$
In each case, we count the number of ways to choose the fixed points and multiply by the number of permutations of the smaller set with no fixed points.  Now $d_{2}$ is the number of permutations of $\{0,1\}$ that have no fixed points.  There are two permutations of  $\{0,1\}$, viz., the identity and the one that swaps the pair.  The swap has no fixed points, so $d_{2} = 1$.  You can see easily by counting that $d_{3}$ is 2 or look at it as $$3\times2 =6 - ({\binom{3}{1} d_{2}+ 1) =  6 -({3} \times{1}+ 1} ) = 2$$
\\\
Now we can write $$d_4 = 24 - (c_1 + c_2 + c_3 + c_4) = 24 - (4 \times 2 + 6 \times 1 + 0 +  1) = 24 - 15 = 9$$
So the probability of a complete ``derangement" of the meals is 9/24 = 3/8.

\section*{Bonus solution}
The approach used above could be extended to provide a recursive formula.  I can't see an easy way to solve the resulting recursion though.  I wonder if someone else can.  The standard way to solve this problem is to use inclusion-exclusion to count the number of permutations that have fixed points (i.e., the ones where at least one diner gets the right meal) and then compute 1 - P(there is a fixed point).  There is a beautiful punch line below that I can't claim credit for nor can I remember where I first saw it.

Let $P$ be the set of all permutations of $n = \{0, ... , n -1 \}$ and let $A \subset P$ be those that have at least one fixed point.  Then for $i = 0,...,n-1$ , define $$A_i = \{p \in P : p(i) = i\}$$ so $A_i$ are the permutations that fix i.  It's easy to see that $$A=\bigcup\limits_{i=0}^{n-1} A_{i}$$ (nice exercise to prove this).

We count the union $A = \bigcup\limits_{i=0}^{n-1} A_{i}$ using  $\href{https://en.wikipedia.org/wiki/Inclusion%E2%80%93exclusion_principle} {\textrm{inclusion / exclusion.}}$
\\
Inclusion-exclusion allows us to count the union of the $A_i$ by starting with $\sum_{i=0}^{n-1} |A_{i}| $ and then correcting it by subtracting cardinalities of pairwise intersections, then correcting that by adding three-way intersection cardinalities, then subtracting.... all the way until we get to the intersection of all of the $A_i$ .
\\
\\
$|A| = \sum_{i=0}^{n-1} |A_{i}| -  \\ 
\indent
\sum_{i < j} {|A_i \cap A_j|} + \\
\indent
\sum_{i < j < k} {|A_i \cap A_j\cap A_k|} - \\
\indent
 ... (-1)^j \sum_{ i_0 < ...<i_{j-1}}{ |A_{i_0} \cap A_{i_1} ... \cap A_{i_{j-1}}|} + ...\\
 \indent
(-1)^{n-1}|A_{0} \cap ... \cap A_{n-1}|$
\\
The notation above requires clarification. The $i<j$ used to define the second sum means that the sum is taken over all possible pairs of sets from the $A_i$.  Similarly, ${i < j < k}$ on the next sum means all three-way intersections, and so on up to the last term, which will be $\pm 1$ times the cardinality of the intersection of all of $A_i$.  In the same way, $ i_0 < ...<i_{j-1}$ refers to an arbitrary $j-$way intersection among the $A_i$ and in this context $j$ is an arbitrary integer between 3 and $n-1$ (inclusive).  Using $<$ in all of the sum bounds makes sure that each distinct combination of indices is counted exactly once.

All of the $A_i$ have the same cardinality.  That cardinality is $(n-1)!$.  The best way to see this is to consider that a permutation of $\{0, ...,  n - 1\}$ that fixes $i$ is an extension of a permutation of the other $n - 1$ elements.  Each distinct permution of the elements other than $i$ extends to exactly one permutation of $n$ that fixes $i$.  So the number of permutations of $n$ that fix a given single element is the same as the number of permutations of $n - 1$.
Therefore,
$$\sum_{i=0}^{n-1} |A_{i}| = n(n-1)!$$
The sum includes $n$ terms, each with the value $(n - 1)!$

Similarly, a permutation of $n$ that fixes a two-element subset is an extension of exactly one permutation of $n-2$, so every pairwise intersection of the $A_i$ has cardinality $()n - 2)!$.  There are $\binom{n}{2} $distinct pairwise intersections among the $n$ sets.  All have cardinality $(n - 2)!$.  Therefore,
$$ \sum_{i < j} {|A_i \cap A_j|}  = \binom{n}{2} {(n - 2)}!$$

In the same way, 
$$\sum_{ i_0 < ...<i_{j-1}}{ |A_{i_0} \cap A_{i_1} ... \cap A_{i_{j-1}}|} = \binom{n}{j}(n - j)!$$

There is exactly one permutation that fixes all of the elements of $n$ (the identity), so last term, 
$$(-1)^{n-1}|A_{0} \cap ... \cap A_{n-1}|$$ 
is either $1$ or $-1$ depending on whether $n$ is odd ($1$) or  even ($-1$).

Putting this all together,  we have $$|A| = \sum_{i=1}^{n}(-1)^{i-1}\binom{n}{i}(n - i)! =  \sum_{i=1}^{n}(-1)^{i-1}\frac{n!}{(n - i)!i!}(n - i)!  = n!  \sum_{i=1}^{n}(-1)^{i-1}/i!$$ 

The solution is therefore $$1 - |A|/n! = 1 - (n!  \sum_{i=1}^{n}(-1)^{i-1}/i)/n! = 1 - \sum_{i=1}^{n}(-1)^{i-1}/i! $$

It is interesting to consider what happens to this probability as $n$ gets large.  Does it go to 0 or 1, converge to some value between or just bounce around?  A beautiful observation that I don't know who to credit for is that the solution above is the $nth$ partial sum of the Taylor Series for $e^x$ evaluated at $x=-1$.  The Taylor series for $e^x$ is $1 + x + x^2 / 2! + x^3 / 3! + ... + x^n/n! + ...$.  So the probabilities converge to $e^{-1} = 1/e.$

\end{document} 