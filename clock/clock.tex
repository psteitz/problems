\documentclass[11pt,a4paper]{report}
\usepackage{amsmath,amsfonts,amssymb,amsthm,epsfig,epstopdf,titling,url,array}
\usepackage{enumitem}
\usepackage{changepage}
\usepackage{graphicx}
\usepackage{caption}
\theoremstyle{plain}
\newtheorem{thm}{Theorem}[section]
\newtheorem{lem}[thm]{Lemma}
\newtheorem{prop}[thm]{Proposition}
\newtheorem*{cor}{Corollary}
\theoremstyle{definition}
\newtheorem{defn}{Definition}[section]
\newtheorem{conj}{Conjecture}[section]
\newtheorem{exmp}{Example}[section]
\newtheorem{exercise}{Exercise}[section]
\theoremstyle{remark}
\newtheorem*{rem}{Remark}
\newtheorem*{note}{Note}
\def\changemargin#1#2{\list{}{\rightmargin#2\leftmargin#1}\item[]}
\let\endchangemargin=\endlist 
\begin{document}

\section*{Problem}
Moe has built an awesome atomic clock.  It has one hand that moves at a perfectly constant speed around a 24 hour dial. The problem is he has no way to calibrate it.  He asks Joe for advice and Joe asks him what his objective is.  He says he wants it to be exactly right as often as possible.  Joe thinks for a minute and says, ``That’s going to take a lot of electricity.'' Explain what he means.

\section*{Solution}
Let $s$ be the rotational speed that Moe chooses expressed in revolutions per day. If Moe can find \emph{exactly} 1 as the rotational speed, he will be right all the time.  Unfortunately, the probability of that is $0$. Now consider what happens when Moe's chosen speed is off by $\delta$ revolutions per day.  If $\delta < 0$ and Moe manages to start the clock at exactly the right time, it will drift by $\delta$ each 24 hours and it will take $1/\delta$ days for it to be exactly right again (when it has fallen so far behind that it is showing exactly the right time). Somewhat paradoxically, the smaller $\delta$ is, the longer this will take and hence the smaller the number of times per any time interval that the clock will be exactly right. The same analysis obviously applies for $\delta > 0$. Now suppose that Moe just lets the hand move as fast as possible. Then it will pass the correct time once every revolution it makes (plus whatever time it takes to complete a revolution). So the faster he can make it go, the more frequently the reading is exactly right. Hence Joe's comment ``that will take a lot of electricity.''
\end{document}



