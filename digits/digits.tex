\documentclass[11pt,a4paper]{report}
\usepackage{amsmath,amsfonts,amssymb,amsthm,epsfig,epstopdf,titling,url,array}
\usepackage{enumitem}
\usepackage{changepage}
\usepackage{graphicx}
\usepackage{caption}
\usepackage{listings}
\usepackage{color}
\theoremstyle{plain}
\newtheorem{thm}{Theorem}[section]
\newtheorem{lem}[thm]{Lemma}
\newtheorem{prop}[thm]{Proposition}
\newtheorem*{cor}{Corollary}
\theoremstyle{definition}
\newtheorem{defn}{Definition}[section]
\newtheorem{conj}{Conjecture}[section]
\newtheorem{exmp}{Example}[section]
\newtheorem{exercise}{Exercise}[section]
\theoremstyle{remark}
\newtheorem*{rem}{Remark}
\newtheorem*{note}{Note}
\def\changemargin#1#2{\list{}{\rightmargin#2\leftmargin#1}\item[]}
\let\endchangemargin=\endlist 

\definecolor{codegreen}{rgb}{0,0.6,0}
\definecolor{codegray}{rgb}{0.5,0.5,0.5}
\definecolor{codepurple}{rgb}{0.58,0,0.82}
\definecolor{backcolour}{rgb}{0.95,0.95,0.92}

\lstdefinestyle{mystyle}{
	backgroundcolor=\color{backcolour},   
	commentstyle=\color{codegreen},
	keywordstyle=\color{magenta},
	numberstyle=\tiny\color{codegray},
	stringstyle=\color{codepurple},
	basicstyle=\footnotesize,
	breakatwhitespace=false,         
	breaklines=true,                 
	captionpos=b,                    
	keepspaces=true,                 
	numbers=left,                    
	numbersep=5pt,                  
	showspaces=false,                
	showstringspaces=false,
	showtabs=false,                  
	tabsize=2
}

\lstset{style=mystyle}
\begin{document}

\section*{Problem}
Suppose that a random integer is generated between 1 and 1000 (inclusive).  What is the probability that one of the digits in the number is a 2? 

\section*{Bonus}
What about the other digits?  Are any of them different from 2?  Which ones and what are their probabilities of occurrence?

\section*{Bonus$^2$}
Write a simple program to verify your answers.

\newpage
\section*{Solution}
There are 271 numbers in the given range that have 2 as one of their digits.  Here is one way to count them.  There is one 1-digit number containing 2.  The 2-digit ones are 20, ... , 29 and 12, ... , 92.  That would make 19, but 22 is counted twice in those lists, so that makes 18 of these.  The 3-digit ones are all of the 200's plus the 2-digit ones preceded by a different digit, with the slight augmentation that in each case, there is one more - the one starting with 0 (e.g. 302).  So there are $100 + 8 \times 19 = 252$ 3-digit numbers containing 2.  So the total is $1 + 18 + 252 = 271.$
Therefore the probability is $271 / 1000 = .271$.
\section*{Bonus Solution}
The same argument above works for 3, 4, ..., 9.  Since 1000 itself includes a 1, that digit is slightly more likely - $272/1000 = .272.$  Now for 0, the counting method above does not work, because 01 is the same as 1, etc.  The 2-digit ones are 10,20,30...,90 and the 3-digit ones are each of these with a 1, up to 9 in front (81 of these) multiplied by 2 (because the 0 could be in either place) plus 100, ..., 900, 1000.  This makes $9 + 2 \times 81 + 10 = 181$ so the probability of 1 is just $.181.$

\pagebreak

\section*{Bonus$^2$ Solution}
Here is some simple Java code to do this:
\begin{lstlisting}[language=Java]
import java.util.Arrays;

public class Counter {
	
	public static void main(String[] args) {
		final int[] counts = new int[10];
		for (int i = 1; i < 1001; i++) {
			for (int j = 0; j < 10; j++) {
				if (contains(i, j)) {
					counts[j]++;
				}
			}
		}
		System.out.println(Arrays.toString(counts));
	}
	
	/**
	* Returns true iff the string representation of y is a substring of
	* the string representation of x.
	*
	* @param x integer to search
	* @param y integer sought as substring
	* @return true if the digits of y occur as a subsequence of the digits of x
	*/
	private static boolean contains(int x, int y) {
		final String xString = String.valueOf(x);
		final String yString = String.valueOf(y);
		return xString.contains(yString);
	}
	
}

\end{lstlisting}

 
\end{document}

