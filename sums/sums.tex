\documentclass[11pt,a4paper]{report}
\usepackage{amsmath,amsfonts,amssymb,amsthm,epsfig,epstopdf,titling,url,array}
\usepackage{enumitem}
\usepackage{changepage}
\usepackage{graphicx}
\usepackage{caption}
\usepackage{listings}
\usepackage{color}
\usepackage{hyperref}
\theoremstyle{plain}
\newtheorem{thm}{Theorem}[section]
\newtheorem{lem}[thm]{Lemma}
\newtheorem{prop}[thm]{Proposition}
\newtheorem*{cor}{Corollary}
\theoremstyle{definition}
\newtheorem{defn}{Definition}[section]
\newtheorem{conj}{Conjecture}[section]
\newtheorem{exmp}{Example}[section]
\newtheorem{exercise}{Exercise}[section]
\theoremstyle{remark}
\newtheorem*{rem}{Remark}
\newtheorem*{note}{Note}
\def\changemargin#1#2{\list{}{\rightmargin#2\leftmargin#1}\item[]}
\let\endchangemargin=\endlist 
\definecolor{codegreen}{rgb}{0,0.6,0}
\definecolor{codegray}{rgb}{0.5,0.5,0.5}
\definecolor{codepurple}{rgb}{0.58,0,0.82}
\definecolor{backcolour}{rgb}{0.95,0.95,0.92}
\hypersetup{
	colorlinks=true,
	linkcolor=blue,
	filecolor=magenta,      
	urlcolor=cyan,
}
\urlstyle{same}
\lstdefinestyle{mystyle}{
	backgroundcolor=\color{backcolour},   
	commentstyle=\color{codegreen},
	keywordstyle=\color{magenta},
	numberstyle=\tiny\color{codegray},
	stringstyle=\color{codepurple},
	basicstyle=\footnotesize,
	breakatwhitespace=false,         
	breaklines=true,                 
	captionpos=b,                    
	keepspaces=true,                 
	numbers=left,                    
	numbersep=5pt,                  
	showspaces=false,                
	showstringspaces=false,
	showtabs=false,                  
	tabsize=2
}
\lstset{style=mystyle}
\begin{document}

\section*{Problem}
Suppose that you have one thousand single dollar bills and you want to put it into a set of envelopes in such a way that any amount up to \$1000 can be made by combining the contents of some subset of the envelopes.  What is the smallest number of envelopes that you need and how would you stuff them?
\\
\\
\textit{Source:} This problem is from \href{https://www.cartalk.com}{Car Talk}

\section*{Solution}
 Use 1, 2, 4, 8, 16, 32, 64, 128, 256 and 489.  To see why this will give all the sums, note that every number has a binary expansion and the expansion of any number up to 511 is a sum of the first nine powers of 2.  To get the rest of the numbers, you can start with 489 and then get the remainder in the same way.  
 \\
 \\
 What is interesting to prove is that this can't be done with fewer than 10 envelopes and this solution is unique.
 

\end{document}

