\documentclass[11pt,a4paper]{report}
\usepackage{amsmath,amsfonts,amssymb,amsthm,epsfig,epstopdf,titling,url,array}
\usepackage{changepage}
\usepackage{graphicx}
\usepackage[utf8]{inputenc}
\theoremstyle{plain}
\newtheorem{thm}{Theorem}[section]
\newtheorem{lem}[thm]{Lemma}
\newtheorem{prop}[thm]{Proposition}
\newtheorem*{cor}{Corollary}
\theoremstyle{definition}
\newtheorem{defn}{Definition}[section]
\newtheorem{conj}{Conjecture}[section]
\newtheorem{exmp}{Example}[section]
\newtheorem{exercise}{Exercise}[section]
\theoremstyle{remark}
\newtheorem*{rem}{Remark}
\newtheorem*{note}{Note}
\begin{document}

\textit{Problem:} Find a function $f(x_0, ..., x_{n-1})$ that maps finite sequences $<x_1, ..., x_{n-1}>$ of positive integers one-to-one into the integers.  From a programmer's perspective, this means find a function that takes a variable number of positive integer parameters and returns a value that is unique to the sequence.  For example, the function $f(x_0, ..., x_{n-1}) = \sum_{i=0}^{n-1}{x_i}$ does not work, since with $f$ defined this way $f(1,2,3)$ returns the same value as $f(3,3)$.

What I am asking you to define here is essentially a \textit{guaranteed unique} hash function.

\textit{Solution (Gödel):} Let $p_0, p_1, ..., p_n, p_{n+1}, ...$ be a list of the (infinitely many) prime numbers in ascending order.  So $p_0 = 2, p_1 = 3$ and so on.  Define 
$$f(x_0, ..., x_{n-1}) = p_0^{x_0}p_1^{x_1}...p_{n-1}^{x_{n-1}}$$
$f$ maps $<x_1, ..., x_{n-1}>$ to the product of the first $n-1$ primes with exponents equal to the corresponding numbers in the sequence.  For example, $f(3,2,1) = 2^{3} \times 3^{2} \times 5^{1} = 380.$
This function is 1-1 because of the uniqueness of prime factorization (the Fundamental Theorem of Arithmetic).  If $f(x_0, ..., x_{n-1}) = f(y_0, ..., y_{m-1})$ then $p_0^{x_0}p_1^{x_1}...p_{n-1}^{x_{n-1}} = p_0^{y_0}p_1^{y_1}...p_{m-1}^{x_{m-1}}$; but this means that both of these expressions are prime factorizations of the same number.  By the uniqueness of prime factorization, it follows that $n = m$ and all of the exponents are the same, which means for all $i = 0, ..., n-1, x_i = y_i$, i.e., the sequences are the same.

\textit{Remark:} Above would actually be a bad hash function because it would overflow very quickly. Hash functions are not guaranteed unique partly because they in general can't be: their range is a finite subset of the positive integers.

\end{document}